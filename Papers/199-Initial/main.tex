%Thesis Proposal for Dec 14, 2015

%\documentclass[runningheads]{llncs}
\documentclass{acm_proc_article-sp}
\setcounter{secnumdepth}{4}
\setcounter{page}{1}
\usepackage{listings}
\usepackage{tabto}


\lstset{
  columns=fullflexible,
}
\usepackage{framed}

% \authorrunning{A. Castillo, J.M.E. Jimenez}
% \titlerunning{(Dec 2015)Implementation of Evolution-Communication P Systems with energy using String Objects in P-Lingua}

\newtheorem{mydef}{Definition}
\newtheorem{mythm}{Theorem}
\newtheorem{mylem}{Lemma}
\newtheorem{myobserve}{Observation}
\newtheorem{mycoro}{Corollary}
\newtheorem{myprop}{Proposition}
%\usepackage{upgreek}

%\newcommand{\PiRS}{\rm \Pi_{\rm RS}}
%\newcommand{\PiBS}{\rm \Pi_{\rm BS}}
%\newcommand{\PiV}[2]{\rm \Pi_{\varphi[{#1,#2}]}}

\usepackage{amsmath,amssymb,latexsym,graphicx,url,listings, algorithm, algorithmic}
\usepackage {ulem,textcomp,lineno}
%\usepackage[table]{xcolor}
%\usepackage{tabularx}
\usepackage{tikz}
\usetikzlibrary{arrows,automata}
%\usetikzlibrary{calc}

%\linenumbers
%\setlength\linenumbersep{5pt}
%\renewcommand\linenumberfont{\normalfont\tiny\sffamily\color{gray}}
%\setpagewiselinenumbers

\usepackage {todonotes}
\usepackage{hyperref,multirow}

%\newcommand{\ra}{\rightarrow}
%\newcommand{\la}{\leftarrow}
%\newcommand{\lra}{\longrightarrow}
%\newcommand{\Ra}{\Rightarrow}
%\newcommand{\Lra}{\Longrightarrow}
\newcommand{\Piu}{\rm\Pi}


%\newcommand{\addnote}[1]{\color{blue} #1 \color{black}}
%\newcommand{\remnote}[2]{\color{red} \sout{#1} \color{blue} #2 \color{black}}
\newcommand{\modnote}[1]{\color{red} #1 \color{black}}

\begin{document}
\title{Implementation of Evolution-Communication P Systems with energy in P-Lingua}
\numberofauthors{1}

	\author{
		\alignauthor{Algina Castillo, Joni Marie E. Jimenez, \\Nestine Hope S. Hernandez, Richelle Ann B. Juayong, Henry N. Adorna}\\
		\affaddr{Algorithms and Complexity Laboratory, Department of Computer Science}\\
		\affaddr{College of Engineering, University of the Philippines-Diliman} \\
		\affaddr{Diliman, Quezon City, Philippines} \\
		{
		\email{\{ancastillo,jejimenez,rbjuayong\}@up.edu.ph}
		\email{\{nshernandez,hnadorna\}@dcs.upd.edu.ph}		}
}
% \title{Implementation of Evolution-Communication P Systems with energy in P-Lingua
% \author{Algina Castillo, Joni Marie E. Jimenez}
% \institute{Algorithms \& Complexity Lab\\
% Department of Computer Science\\https://preview.overleaf.com/public/pkwkrbhttpqy/images/a5a302fe9c3fb3f66cba092e51ed4fc2006d6818.jpeg
% University of the Philippines Diliman\\
% Diliman 1101 Quezon City, Philippines;\\
% \href{ancastillo@up.edu.ph}{ancastillo@up.edu.ph},
% \href{mailto:jejimenez@up.edu.ph}{jejimenez@up.edu.ph},
% }
% }

\maketitle

%%%%%%%%%%%%%%%%%%%%%%%%%%%%%%%%%%%%%%%%%%%%%%%%%%%%%%%%%%%%%%%%%%%
\begin{abstract}
		In this proposal, we present our initial research regarding the implementation of Evolution-Communication P system with energy (ECPe systems) in P-Lingua
       ECPe Systems has been proposed as an extension to Evolution-Communication P Systems. The rules in ECPe works exactly as in ECP Systems except for the additional consideration for energy. Energy is a property of the regions and is represented by an integer greater than or equal to 0. Initial configuration does not permit having an energy in the regions. Instead it is produced during evolution and is consumed during communication.
\indent A construct for ECPe had been proposed already. ECP Systems had been implemented in P-Lingua. This paper will propose an implementation of ECPe Systems in P-Lingua by extending ECP Systems in P-Lingua.
       \end{abstract}

\noindent {\bf Key words:} Membrane computing; Evolution-Communication P System with energy; PLingua
    
%%%%%%%%%%%%%%%%%%%%%%%%%%%%%%%%%%%%%%%%%%%%%%%%%%%%%%%%%%%%%%%%%%%
%%%%%%%%%%%%%%%%%%%%%%%%%%%%%%%%%%%%%%%%%%%%%%%%%%%%%%%%%%%%%%%%%%%
\section{Introduction}
\indent\textit{Membrane computing}\cite{p-system} is a branch of natural computing aiming to abstract computing ideas and models from the structure and the functioning of living cells. Membrane Computing deals with distributed and parallel computing models called P Systems which processes multisets of symbol objects in a localized manner with an essential role played by communication between compartments. Many different types of P Systems exist.\\
\indent \textit{ECP Systems with energy}\cite{ecpe_no-antiport} is a variant of P Systems that originates from Evolution-Communication P Systems. Evolution-Communication P Systems, on the other hand, is a hybrid of Transition P Systems and P Systems with Symport and Antiport rules. ECPe Systems has a special object called 'energy' that is produced in evolution rules and consumed in communication rules.\\
\indent \textit{P-Lingua} \cite{psystem-website} is a programming language used to compile and simulate P systems. As of this time of writing, P-Lingua and its library- pLinguaCore is in its version 4.0 and already has a compiler and simulator for Spiking Neural P Systems, Kernel P Systems, Tissue-Like P Systems with Cell separation rules and with division rules, Population Dynamics P systems , Active Membrane P Systems with division rules, Transition P Systems, and many other cell-like P systems.

\section{Evolution Communication P-Systems with energy (ECPe)}
 ECPe systems have also been implemented in GPU in a form of matrix.
     
    Evolution Communication P systems with energy has the following construct: \cite{ecpe_no-antiport}
    	\begin{center}
        $\prod = (O, $e$, \mu,\omega_i,...,\omega_m,R_1,R'_1,...,R_m,R'_m,i_{out})$\\
        	\end{center}
        where:
        	\begin{itemize}
            	\item $m$ - the total number of membranes 
				\item $O$ - alphabet of objects
                \item $\mu$ - membrane structure
      			\item $\omega_1,...,\omega_m$ - strings over $O^*$, where $\omega_i$ denotes          the multiset of object present in the region bounded by membrane i.
        		\item $i_{out}$ - output membrane
				\item $R_1,...,R_m$ - set of evolution rules. Object $e$ can be produced, but should never be in the initial configuration and is not allowed to evolve.
				\item $R'_1,...,R'_m$ - set of communication rules. Symport has rules ($ae^i,in$) or ($ae^i,out$), where $a \in 0$, $i\geq 1$. Antiport has rule ($ae^i,out;be^j,in$), $i,j\geq 1$, where $a,b \in O$, $i,j \geq 1$. Copies of object $e$ are lost after application.   
			\end{itemize}

\section{P-Lingua} 
	  \indent P-Lingua is a programming language for Membrane Computing \cite{psystem-website}. Its library \textit{pLinguaCore}, which is written in Java and now on its version 4.0, can be downloaded and edited under GPL license from their website\cite{psystem-website}  .  \\
       \indent At this time of writing, P-Lingua can already compile and simulate Spiking Neural P Systems, Kernel P Systems, Tissue-Like P Systems with Cell separation rules and with division rules, Population Dynamics P systems , Active Membrane P Systems with division rules, and many other cell-like P systems.\\
       \indent For this paper, let's look into the models (originally presented in  \cite {p-system}) of P-Lingua that might be helpful.
  
       \subsubsection{Model.}
        \indent  Since P-Lingua supports many P systems, it is necessary to identify what particular P system is our file defined.This is important because not each type of rule is allowed in every model, for example, membrane creation rules are not permitted in P systems with symport/antiport rules. The built-in compiler of P-Lingua detects such error. Models
are specified by using $@model<model name>$ and at this stage, the allowed
models are:\\
$@model<membrane\_division> $\\
$@model<membrane\_creation>$\\
$@model<transition>$\\
$@model<probabilistic>$\\
$@model<evolution\_communication>$\\
$@model<ev\_symport\_antiport>$\\
$@model<symport\_antiport>$\\
$@model<TSCS>$\\
$@model<spiking\_psystems>$\\
$@model<tissue\_psystems>$
 
	\subsubsection{ Rules.} For relevance and simplicity, let's have a look on the rules in the model for ECP Systems which is $ @model<evolution\_communication>$. \\
        \subsubsection*{Evolution Rule.}
        
\indent     $[a --> b]'h$ \\where $a$ and $b \in O$ and $h$ is the label of the membrane wherein this evolution rule is applicable.
     \subsubsection*{Send-in Rule.}
        
\indent     $a[]'h-->[a]$ \\where $b \in O$, and $h$ is the label of the child membrane of the membrane containing $a$.$a$ will be moved to membrane with label $h$ after rule application.
     \subsubsection*{Send-out Rule.}
        
\indent     $[b]'h --> b[]$ \\where $b \in O$, and $h$ is the label of the membrane wherein $a$ is located.$a$ is then moved to the parent membrane of membrane $h$ after rule application.
     \subsubsection*{Antiport Rule.}
        
\indent     $\ a[b]'h--> b[a]'h$ \\where $a$ and $b \in O$,and $h$ is the label of the membrane where $b$ is contained before the rule is applied.

\subsection{Model for ECPe Systems}
	\indent ECPe Systems have been extended in ECP, thus, also has a model name of $"evolution\_communication"$.\\
   
	\indent ECPe copies the same syntax as ECP Systems for function definition, membrane configuration initialization, and multiset initialization for each membrane. It can do so because of the absence of the energy $e$ during the initialization. ECP and ECPe syntax will differ during the rule implementation, which we will now show:  \\
    Evolution rules
    	\begin{center}
			[ $a$ ]'$h$ - -$>$ [ $b$ ] : $e$\\
		\end{center}            
            where: \\
        
         \begin{itemize}
			\item $a$ and $b$ $\in O.$
            \item $h$ is a label of the membrane where the evolution rule applies
            \item $e$ is an integer $\geq$ 0. It is represented by the energy produced during evolution.
		\end{itemize}

	Communication rules: Symport\\
    	For send-in:\\
        \begin{center}
            	$a$[ ]'$h$:$e$ - -$>$ [$a$]
         \end{center}
         	where: \\
         	\begin{itemize}
				\item $a \in O.$ 
                \item $h$ is the membrane label where $a$ will be sent in. $a$ will always be in the membrane immediately outside membrane $h$. After the rule is applied, $a$ will be inside membrane $h$. 
               	\item $e$ is an integer $\geq$ 1. It is represented by the energy consumed during communication. The rule cannot be applied without the sufficient number of $e$ in the membrane directly outside of $h$.
            \end{itemize}

         For send out:\\
         \begin{center}
			[$b$]'$h$:$e$ - - $>$ $b$[ ]     
		\end{center}
        where: \\
        \begin{itemize}
				\item $b \in O.$ 
                \item $h$ is the label of the membrane where $b$ is contained initially. After the rule is applied, $a$ will be transferred the membrane immediately outside $h$ .
                \item $e$ is an integer $\geq$ 1. It is represented by the energy consumed during communication. The rule cannot be applied without the sufficient number of $e$ in membrane $h$.
            \end{itemize}
            
	Communication rules: Antiport 
    	\begin{center}
			$b$[$a$]'$h$:$e_1$ - - $>$ $a$[$b$]'$h$:$e_2$ 
		\end{center}
        where:\\
        	\begin{itemize}
				\item $a$ and $b$ $\in O.$
                \item $h$ is the label of the membrane where $a$ is contained initially. After the rule is applied, $a$ will be transferred the membrane immediately outside $h$. Consequently, $b$, which is in the membrane outside the membrane $h$, will be transferred inside the membrane $h$.
				\item $e_1, e_2 \geq 1$. $e_1$ is associated to the membrane where $a$ is initially contained (membrane $h$) while $e_2$ is associated to the membrane where $b$ is initially contained (membrane outside $h$). The rule cannot be applied without sufficient number of $e_1$ in the region of membrane $h$ and without sufficient number $e_2$ in the region directly outside membrane $h$.
            \end{itemize}    

\section{P-Lingua modification}
	Since ECPe is an extension of ECP, we decided to extend the model for ECP which is $<evolution\_communication>$ in P-Lingua to include the implementation of energy $e$. Due to this, we didn't add any additional classes and directories in the P-Lingua Java. Firstly, in order to recognize $e$ in P-Lingua input (indicated by a colon), we modified the Parser to incorporate the use of energy. Though it was successful, this resulted in the lost of the use of ranges. Ranges are another component in P-Lingua that uses colon as an indicator.\\
    After taking care of the input, we made the energy an attribute of a membrane - both of the configuration and the rule. Then we added the energy attribute in the compiled output.\\
    As for the simulation, we modified the classes responsible for executing cellLike simulation to incorporate the use of energy. For example, adding energies during implementation of evolution rule and subracting energies during communication rule from involved membranes.
    
  %Specifically, we added energy in the constructors of RightHandrule.java, HandRule.java, and made the energy accessible inside CellLikeSkinMembrane.java and CellLikeSkinNoMembrane.java. Finally, we set the behavior of energy during simulation inside the CellLikeRule.java. Here, we checked if the energy is an integer greater than or equal to 0.  Also, we checked if the energy is sufficient to allow a communication rule, and executed the rule if it is. We added energies during evolution and subtracted during communication.\\

\section{Testing and Results}

\subsection{Sample Input}
$@model<evolution\_communication>$ \\
def SAMP() $\lbrace$ \\
$@mu = [ [ [ [ ]'4]'3]'2]'1;$\\
$@ms(4) = ab,s,c;$\\
$@ms(3) = bad,c;$\\
$@ms(2) = fg,ab;$\\
$@ms(1) = bad,ab*2;$\\\\
$[bad]'1 --> [cd]:1;$\\
$[ab]'1 --> [fg]:1;$\\
$cd[]'2:4--> [cd]'2;$\\
$fg[]'3:2 --> [fg];$\\\\
$[ab]'3-->[fg]:3;$\\
$c[s]'3:2 --> s[c]'3:1;$\\\\ 
$[bad]'3-->[s]:3;$\\
$[s]'4--> [bd]:5;$\\
$[bd]'4:1 --> bd[];$\\
$\rbrace$\\\\
def main() $\lbrace$\\
$\indent$ call SAMP();\\
$\rbrace$

\subsubsection{Simulation Output}

\begin{lstlisting}

***********************************************

    CONFIGURATION: 0
    TIME: 0.0 s.
    MEMORY USED: 62976
    FREE MEMORY: 57060
    TOTAL MEMORY: 932352

    MEMBRANE ID: 3, Label: 4, Charge: 0, Energy: 0
    Multiset: ab, s, c
    Parent membrane ID: 2

    MEMBRANE ID: 2, Label: 3, Charge: 0, Energy: 0
    Multiset: bad, c
    Internal membranes count: 1
    Parent membrane ID: 1

    MEMBRANE ID: 1, Label: 2, Charge: 0, Energy: 0
    Multiset: fg, ab
    Internal membranes count: 1
    Parent membrane ID: 0

    SKIN MEMBRANE ID: 0, Label: 1, Charge: 0, Energy: 0
    Multiset: bad, ab*2
    Internal membranes count: 1

-----------------------------------------------

    STEP: 1

    Rules selected for MEMBRANE ID: 3, Label: 4, Charge: 0, 
    Energy: 5
    1 * #-1 [s --> bd]'4

    Rules selected for MEMBRANE ID: 2, Label: 3, Charge: 0, 
    Energy: 3
    1 * #-1 [bad --> s]'3

    Rules selected for SKIN MEMBRANE ID: 0, Label: 1, 
    Charge: 0, Energy: 2
    2 * #-1 [ab --> fg]'1
    1 * #-1 [bad --> cd]'1

***********************************************

    CONFIGURATION: 1
    TIME: 0.162 s.
    MEMORY USED: 62976
    FREE MEMORY: 56405
    TOTAL MEMORY: 932352

    MEMBRANE ID: 3, Label: 4, Charge: 0, Energy: 5
    Multiset: ab, c, bd
    Parent membrane ID: 2

    MEMBRANE ID: 2, Label: 3, Charge: 0, Energy: 3
    Multiset: c, s
    Internal membranes count: 1
    Parent membrane ID: 1

    MEMBRANE ID: 1, Label: 2, Charge: 0, Energy: 0
    Multiset: fg, ab
    Internal membranes count: 1
    Parent membrane ID: 0

    SKIN MEMBRANE ID: 0, Label: 1, Charge: 0, Energy: 2
    Multiset: fg*2, cd
    Internal membranes count: 1

------------------------------------
-----------

    STEP: 2

    Rules selected for MEMBRANE ID: 3, Label: 4, Charge: 0, 
    Energy: 4
    1 * #-1 [bd]'4:1 --> []bd:0

***********************************************

    CONFIGURATION: 2
    TIME: 0.174 s.
    MEMORY USED: 62976
    FREE MEMORY: 56405
    TOTAL MEMORY: 932352

    MEMBRANE ID: 3, Label: 4, Charge: 0, Energy: 4
    Multiset: ab, c
    Parent membrane ID: 2

    MEMBRANE ID: 2, Label: 3, Charge: 0, Energy: 3
    Multiset: c, s, bd
    Internal membranes count: 1
    Parent membrane ID: 1

    MEMBRANE ID: 1, Label: 2, Charge: 0, Energy: 0
    Multiset: fg, ab
    Internal membranes count: 1
    Parent membrane ID: 0

    SKIN MEMBRANE ID: 0, Label: 1, Charge: 0, Energy: 2
    Multiset: fg*2, cd
    Internal membranes count: 1

Halting configuration 
(No rule can be selected to be executed in the next step)
\end{lstlisting}


\subsubsection{Analysis}
The time it took is around 1.79E-4 s.\\
% \noindent MEMORY:
\subsection{Sample Input 2}
\begin{lstlisting}

@model<evolution_communication>
def SAMP() {
@mu = [ [ ]'2]'1;
@ms(2) = S;


[S]'2--> [Sp,a]:2;
[Sp]'2--> [Spp,b];

[a]'2:1-->a[];
[b]'2:1-->b[];
[Spp]'2:1-->Spp[];


[a]'1--> [ap]:1;
[b]'1--> [bp]:1;

[ap]'1:1-->ap[];
[bp]'1:1-->bp[];

}

def main(){
	call SAMP();
}
\end{lstlisting}
\subsubsection{Simulated Output}
\begin{lstlisting}

***********************************************

    CONFIGURATION: 0
    TIME: 0.0 s.
    MEMORY USED: 62976
    FREE MEMORY: 57060
    TOTAL MEMORY: 932352

    MEMBRANE ID: 1, Label: 2, Charge: 0, Energy: 0
    Multiset: S
    Parent membrane ID: 0

    SKIN MEMBRANE ID: 0, Label: 1, Charge: 0, Energy: 0
    Multiset: #
    Internal membranes count: 1

--------------------------------
---------------

    STEP: 1

    Rules selected for MEMBRANE ID: 1, Label: 2, Charge: 0, 
    Energy: 2
    1 * #-1 [S --> Sp, a]'2

***********************************************

    CONFIGURATION: 1
    TIME: 0.095 s.
    MEMORY USED: 62976
    FREE MEMORY: 56732
    TOTAL MEMORY: 932352

    MEMBRANE ID: 1, Label: 2, Charge: 0, Energy: 2
    Multiset: Sp, a
    Parent membrane ID: 0

    SKIN MEMBRANE ID: 0, Label: 1, Charge: 0, Energy: 0
    Multiset: #
    Internal membranes count: 1

-----------------------------------
------------

    STEP: 2

    Rules selected for MEMBRANE ID: 1, Label: 2, Charge: 0, 
    Energy: 1
    1 * #-1 [Sp --> Spp, b]'2
    1 * #-1 [a]'2:1 --> []a:0

***********************************************

    CONFIGURATION: 2
    TIME: 0.109 s.
    MEMORY USED: 62976
    FREE MEMORY: 56405
    TOTAL MEMORY: 932352

    MEMBRANE ID: 1, Label: 2, Charge: 0, Energy: 1
    Multiset: Spp, b
    Parent membrane ID: 0

    SKIN MEMBRANE ID: 0, Label: 1, Charge: 0, Energy: 0
    Multiset: a
    Internal membranes count: 1

--------------------------------
---------------

    STEP: 3

    Rules selected for MEMBRANE ID: 1, Label: 2, Charge: 0, 
    Energy: 0
    1 * #-1 [Spp]'2:1 --> []Spp:0
    1 * #-1 [b]'2:1 --> []b:0

    Rules selected for SKIN MEMBRANE ID: 0, Label: 1, 
    Charge: 0, Energy: 1
    1 * #-1 [a --> ap]'1

***********************************************

    CONFIGURATION: 3
    TIME: 0.118 s.
    MEMORY USED: 62976
    FREE MEMORY: 56405
    TOTAL MEMORY: 932352

    MEMBRANE ID: 1, Label: 2, Charge: 0, Energy: 0
    Multiset: b
    Parent membrane ID: 0

    SKIN MEMBRANE ID: 0, Label: 1, Charge: 0, Energy: 1
    Multiset: Spp, ap
    Internal membranes count: 1

-------------------------------
----------------

    STEP: 4

    Rules selected for SKIN MEMBRANE ID: 0, Label: 1, 
    Charge: 0, Energy: 1
    1 * #-1 [ap]'1:1 --> []ap:0

***********************************************

    CONFIGURATION: 4
    TIME: 0.125 s.
    MEMORY USED: 62976
    FREE MEMORY: 56077
    TOTAL MEMORY: 932352

    MEMBRANE ID: 1, Label: 2, Charge: 0, Energy: 0
    Multiset: b
    Parent membrane ID: 0

    SKIN MEMBRANE ID: 0, Label: 1, Charge: 0, Energy: 1
    Multiset: Spp
    Internal membranes count: 1

    ENVIRONMENT: ap

Halting configuration 
(No rule can be selected to be executed in the next step)

\end{lstlisting}

\subsection{Sample Input 3}
This is modified version of Sample Input 2. A rule is deleted to always end with a halting configuration.
\begin{lstlisting}

@model<evolution_communication>
def SAMP() {
@mu = [ [ ]'2]'1;
@ms(2) = S;

[S]'2--> [S,a]:2;
[S]'2--> [Sp];
[Sp]'2--> [Spp,b];
[Sp]'2--> [Spp];

[a]'2:1-->a[];
[b]'2:1-->b[];
[Spp]'2:1-->Spp[];


[a]'1--> [ap]:1;
[b]'1--> [bp]:1;

[ap]'1:1-->ap[];
[bp]'1:1-->bp[];

}

def main(){
	call SAMP();
}

\end{lstlisting}

\subsubsection{Simulation Output}
\begin{lstlisting}


***********************************************

    CONFIGURATION: 0
    TIME: 0.0 s.
    MEMORY USED: 61440
    FREE MEMORY: 57887
    TOTAL MEMORY: 912064

    MEMBRANE ID: 1, Label: 2, Charge: 0, Energy: 0
    Multiset: S
    Parent membrane ID: 0

    SKIN MEMBRANE ID: 0, Label: 1, Charge: 0, Energy: 0
    Multiset: #
    Internal membranes count: 1

-----------------------------------------------

    STEP: 1

    Rules selected for MEMBRANE ID: 1, Label: 2, Charge: 0, 
    Energy: 2
    1 * #1 [S --> S, a]'2

***********************************************

    CONFIGURATION: 1
    TIME: 0.256 s.
    MEMORY USED: 61440
    FREE MEMORY: 57887
    TOTAL MEMORY: 912064

    MEMBRANE ID: 1, Label: 2, Charge: 0, Energy: 2
    Multiset: S, a
    Parent membrane ID: 0

    SKIN MEMBRANE ID: 0, Label: 1, Charge: 0, Energy: 0
    Multiset: #
    Internal membranes count: 1

---------------------------------
--------------

    STEP: 2

    Rules selected for MEMBRANE ID: 1, Label: 2, Charge: 0, 
    Energy: 3
    1 * #1 [S --> S, a]'2
    1 * #5 [a]'2:1 --> []a:0

***********************************************

    CONFIGURATION: 2
    TIME: 0.529 s.
    MEMORY USED: 61440
    FREE MEMORY: 57887
    TOTAL MEMORY: 912064

    MEMBRANE ID: 1, Label: 2, Charge: 0, Energy: 3
    Multiset: a, S
    Parent membrane ID: 0

    SKIN MEMBRANE ID: 0, Label: 1, Charge: 0, Energy: 0
    Multiset: a
    Internal membranes count: 1

-----------------------------------------------

    STEP: 3

    Rules selected for MEMBRANE ID: 1, Label: 2, Charge: 0, 	Energy: 4
    1 * #1 [S --> S, a]'2
    1 * #5 [a]'2:1 --> []a:0

    Rules selected for SKIN MEMBRANE ID: 0, Label: 1, 
    Charge: 0, Energy: 1
    1 * #8 [a --> ap]'1

***********************************************

    CONFIGURATION: 3
    TIME: 0.807 s.
    MEMORY USED: 61440
    FREE MEMORY: 57566
    TOTAL MEMORY: 912064

    MEMBRANE ID: 1, Label: 2, Charge: 0, Energy: 4
    Multiset: a, S
    Parent membrane ID: 0

    SKIN MEMBRANE ID: 0, Label: 1, Charge: 0, Energy: 1
    Multiset: a, ap
    Internal membranes count: 1

---------------------------------------
--------

    STEP: 4

    Rules selected for MEMBRANE ID: 1, Label: 2, Charge: 0, 
    Energy: 5
    1 * #5 [a]'2:1 --> []a:0
    1 * #1 [S --> S, a]'2

    Rules selected for SKIN MEMBRANE ID: 0, Label: 1, 
    Charge: 0, Energy: 1
    1 * #8 [a --> ap]'1
    1 * #10 [ap]'1:1 --> []ap:0

***********************************************

    CONFIGURATION: 4
    TIME: 1.135 s.
    MEMORY USED: 61440
    FREE MEMORY: 57566
    TOTAL MEMORY: 912064

    MEMBRANE ID: 1, Label: 2, Charge: 0, Energy: 5
    Multiset: a, S
    Parent membrane ID: 0

    SKIN MEMBRANE ID: 0, Label: 1, Charge: 0, Energy: 1
    Multiset: a, ap
    Internal membranes count: 1

    ENVIRONMENT: ap

-----------------------------------------------

    STEP: 5

    Rules selected for MEMBRANE ID: 1, Label: 2, Charge: 0, 
    Energy: 4
    1 * #5 [a]'2:1 --> []a:0
    1 * #2 [S --> Sp]'2

    Rules selected for SKIN MEMBRANE ID: 0, Label: 1, 
    Charge: 0, Energy: 1
    1 * #8 [a --> ap]'1
    1 * #10 [ap]'1:1 --> []ap:0

***********************************************

    CONFIGURATION: 5
    TIME: 1.525 s.
    MEMORY USED: 61440
    FREE MEMORY: 57244
    TOTAL MEMORY: 912064

    MEMBRANE ID: 1, Label: 2, Charge: 0, Energy: 4
    Multiset: Sp
    Parent membrane ID: 0

    SKIN MEMBRANE ID: 0, Label: 1, Charge: 0, Energy: 1
    Multiset: a, ap
    Internal membranes count: 1

    ENVIRONMENT: ap*2

---------------------------------
--------------

    STEP: 6

    Rules selected for MEMBRANE ID: 1, Label: 2, Charge: 0, 
    Energy: 4
    1 * #3 [Sp --> Spp, b]'2

    Rules selected for SKIN MEMBRANE ID: 0, Label: 1, 
    Charge: 0, Energy: 1
    1 * #10 [ap]'1:1 --> []ap:0
    1 * #8 [a --> ap]'1

***********************************************

    CONFIGURATION: 6
    TIME: 1.857 s.
    MEMORY USED: 61440
    FREE MEMORY: 57244
    TOTAL MEMORY: 912064

    MEMBRANE ID: 1, Label: 2, Charge: 0, Energy: 4
    Multiset: Spp, b
    Parent membrane ID: 0

    SKIN MEMBRANE ID: 0, Label: 1, Charge: 0, Energy: 1
    Multiset: ap
    Internal membranes count: 1

    ENVIRONMENT: ap*3

------------------------------------
-----------

    STEP: 7

    Rules selected for MEMBRANE ID: 1, Label: 2, Charge: 0, 
    Energy: 2
    1 * #7 [Spp]'2:1 --> []Spp:0
    1 * #6 [b]'2:1 --> []b:0

    Rules selected for SKIN MEMBRANE ID: 0, Label: 1, 
    Charge: 0, Energy: 0
    1 * #10 [ap]'1:1 --> []ap:0

***********************************************

    CONFIGURATION: 7
    TIME: 2.154 s.
    MEMORY USED: 61440
    FREE MEMORY: 57244
    TOTAL MEMORY: 912064

    MEMBRANE ID: 1, Label: 2, Charge: 0, Energy: 2
    Multiset: #
    Parent membrane ID: 0

    SKIN MEMBRANE ID: 0, Label: 1, Charge: 0, Energy: 0
    Multiset: Spp, b
    Internal membranes count: 1

    ENVIRONMENT: ap*4

-----------------------------------------------

    STEP: 8

    Rules selected for SKIN MEMBRANE ID: 0, Label: 1, 
    Charge: 0, Energy: 1
    1 * #9 [b --> bp]'1

***********************************************

    CONFIGURATION: 8
    TIME: 2.373 s.
    MEMORY USED: 61440
    FREE MEMORY: 57244
    TOTAL MEMORY: 912064

    MEMBRANE ID: 1, Label: 2, Charge: 0, Energy: 2
    Multiset: #
    Parent membrane ID: 0

    SKIN MEMBRANE ID: 0, Label: 1, Charge: 0, Energy: 1
    Multiset: Spp, bp
    Internal membranes count: 1

    ENVIRONMENT: ap*4

-----------------------------------
------------

    STEP: 9

    Rules selected for SKIN MEMBRANE ID: 0, Label: 1, 
    Charge: 0, Energy: 0
    1 * #11 [bp]'1:1 --> []bp:0

***********************************************

    CONFIGURATION: 9
    TIME: 2.607 s.
    MEMORY USED: 61440
    FREE MEMORY: 56923
    TOTAL MEMORY: 912064

    MEMBRANE ID: 1, Label: 2, Charge: 0, Energy: 2
    Multiset: #
    Parent membrane ID: 0

    SKIN MEMBRANE ID: 0, Label: 1, Charge: 0, Energy: 0
    Multiset: Spp
    Internal membranes count: 1

    ENVIRONMENT: ap*4, bp

Halting configuration 
(No rule can be selected to be executed in the next step)


\end{lstlisting}

\section{Conclusion and Future Work}
	This paper tried to add energy in ECP in P-Lingua. After defining the syntax of ECPe, modification in P-Lingua were made. 
    In the future, modification may be needed in the Parser to incorporate both ranges and energy.
    
   
%LaTeX and BibTex compilation/typesetting warnings acceptable, but not errors!!

%The bibliography style! referencing file splncs.bst
\bibliographystyle{splncs}
%the BibTeX file, referencing file snp.bib
\bibliography{snp} 

%LaTeX and BibTex compilation/typesetting warnings acceptable, but not errors!!

\end{document}